%First Beamer Talk

\documentclass{beamer}

% Setup appearance:

\usetheme{Darmstadt}
%\usecolortheme{lily}
%\usefonttheme[onlylarge]{structurebold}
%\setbeamerfont*{frametitle}{size=\normalsize,series=\bfseries}
%\useinnertheme{circles}


% Standard packages

\usepackage[english]{babel}
\usepackage[latin1]{inputenc}
%\usepackage{times}
\usepackage[T1]{fontenc}
\usepackage{multirow}
\usepackage{capt-of}
\usepackage{graphicx}
\usepackage{array}
\usepackage{tikz}

\setbeamertemplate{blocks}[rounded] [shadow=false]

% Some optional colors. Change or add as you see fit.
%---------------------------------------------------
 \definecolor{ualbertagreen}{HTML}{007C41}
\definecolor{ualbertagold}{HTML}{FFDB05}


% Some optional color adjustments to Beamer. Change as you see fit.
%------------------------------------------------------------------
\setbeamercolor{frametitle}{fg=ualbertagreen,bg=white}
\setbeamercolor{title}{fg=ualbertagreen,bg=white}
\setbeamercolor{author}{fg=ualbertagreen,bg=white}
\setbeamercolor{date}{fg=ualbertagreen,bg=white}
\setbeamercolor{affiliation}{fg=ualbertagreen,bg=white}
\setbeamercolor{institute}{fg=ualbertagreen,bg=white}
\setbeamercolor{local structure}{fg=ualbertagreen}
\setbeamercolor{section in toc}{fg=ualbertagreen,bg=white}
% \setbeamercolor{subsection in toc}{fg=ualbertagreen,bg=white}
\setbeamercolor{footline}{fg=ualbertagreen!50, bg=white}
\setbeamercolor{block title}{fg=ualbertagreen,bg=white}
\setbeamercolor{upper separation line head}{bg=ualbertagreen}
\setbeamercolor{lower separation line head}{bg=ualbertagold}
\setbeamercolor{middle separation line head}{bg=ualbertagold}
\setbeamercolor{frametitle}{fg=ualbertagreen,bg=white}

\setbeamercolor{section in head/foot}{bg=white,fg=ualbertagreen}
\setbeamercolor{author in head/foot}{bg=white,fg=ualbertagreen}
\setbeamercolor{date in head/foot}{bg=white,,fg=ualbertagreen}
\setbeamercolor{title in head/foot}{bg=white,fg=ualbertagreen}

\setbeamercolor{headline}{bg=white,fg=ualbertagreen}




\setbeamercolor{middle separation line head}{bg=ualbertagreen}
\setbeamercolor{alerted text}{fg=red}
\setbeamercolor{example text}{fg=black}
\setbeamercolor{structure}{fg=black}

% Various cosmetic things, though I must confess I forget what exactly these do and why I included them.
%-------------------------------------------------------------------------------------------------------
\setbeamercolor{structure}{fg=ualbertagreen}


\setbeamercolor{local structure}{parent=structure}
\setbeamercolor{item projected}{parent=item,use=item,fg=ualbertagreen,bg=white}
\setbeamercolor{enumerate item}{parent=item}

\setbeamertemplate{title page}{%
  \vbox{}
%  \vfill
    \vspace{0cm}% NEW
  \begingroup
    \centering
    \begin{beamercolorbox}[sep=8pt,center]{title}
      \usebeamerfont{title}\inserttitle\par%
      \ifx\insertsubtitle\@empty%
      \else%
        \vskip0.05em%
        {\usebeamerfont{subtitle}\usebeamercolor[fg]{subtitle}\insertsubtitle\par}%
      \fi%
    \end{beamercolorbox}%
    \vskip1em\par
    \begin{beamercolorbox}[sep=8pt,center]{author}
      \usebeamerfont{author}\insertauthor
    \end{beamercolorbox}
    \begin{beamercolorbox}[sep=8pt,center]{institute}
      \usebeamerfont{institute}\insertinstitute
    \end{beamercolorbox}
    \vspace{0.5cm}% NEW
    \begin{beamercolorbox}[sep=8pt,center]{date}
      \usebeamerfont{date}\insertdate
    \end{beamercolorbox}\vskip0.05em
%    {\usebeamercolor[fg]{titlegraphic}\inserttitlegraphic\par}
  \endgroup
%  \vfill
}


\logo{
   \tikz [remember picture,overlay]
    \node[yshift=.3cm,xshift=1.5cm] at (current page.south west)
        %or: (current page.center)
        {\includegraphics[width=1in]{UA-ASB-COLOUR.png}};
%\includegraphics[height=0.8cm]{UA-ASB-COLOUR.png}\vspace{220pt}
}




\setbeamertemplate{headline}{%
\leavevmode%
  \hbox{%
    \begin{beamercolorbox}[wd=\paperwidth,ht=5ex,dp=1.825ex]{white}%
    \usebeamerfont{headline}\hskip6pt\inserttitle\par%
    \insertsectionnavigationhorizontal{\paperwidth}{}{\hskip0pt plus1filll}
    \end{beamercolorbox}%
  }
}

\setbeamertemplate{sidebar right}{}


\logo{
   \tikz [remember picture,overlay]
    \node[yshift=.3cm,xshift=1.5cm] at (current page.south west)
        %or: (current page.center)
        {\includegraphics[width=1in]{UA-ASB-COLOUR.png}};
%\includegraphics[height=0.8cm]{UA-ASB-COLOUR.png}\vspace{220pt}
}




%\setbeamertemplate{footline}{%
%\hfill\usebeamertemplate***{navigation symbols}
%\hspace{1cm}\insertframenumber{}/\inserttotalframenumber}
\defbeamertemplate*{footline}{my footline}{%
    \ifnum\insertpagenumber=1
        \Tiny{%
            \hfill%
		\vspace*{1pt}%
            %\insertframenumber/\inserttotalframenumber \hspace*{0.1cm}%
            \newline%
            \color{ualbertagold}{\rule{\paperwidth}{0.4mm}}\newline%
            \color{ualbertagold}{\rule{\paperwidth}{.4mm}}%
        }
%    \hbox{%
%        \begin{beamercolorbox}[wd=\paperwidth,ht=.8ex,dp=1ex,center]{}%
%      % empty environment to raise height
%        \end{beamercolorbox}%
%    %}%
    %\vskip0pt%
    %no page number on the first page
    %    \Tiny{%
    %        \hfill%
   % 		\vspace*{1pt}%
    %        \color{ualbertagold}{\rule{\paperwidth}{0.4mm}}\newline%
    %        \color{ualbertagold}{\rule{\paperwidth}{.4mm}}%
%        }%
  \else%
        \Tiny{%
            \hfill%
		\vspace*{1pt}%
            \insertframenumber/\inserttotalframenumber \hspace*{0.1cm}%
            \newline%
            \color{ualbertagold}{\rule{\paperwidth}{0.4mm}}\newline%
            \color{ualbertagold}{\rule{\paperwidth}{.4mm}}%
        }%
    \fi%
}




\usepackage[style=british]{csquotes}

\def\signed #1{{\leavevmode\unskip\nobreak\hfil\penalty50\hskip1em
  \hbox{}\nobreak\hfill #1%
  \parfillskip=0pt \finalhyphendemerits=0 \endgraf}}

\newsavebox\mybox
\newenvironment{aquote}[1]
  {\savebox\mybox{#1}\begin{quote}}
  {\vspace*{1mm}\signed{\usebox\mybox}\end{quote}}






\renewcommand{\(}{\begin{columns}}
\renewcommand{\)}{\end{columns}}
\newcommand{\<}[1]{\begin{column}{#1}}
\renewcommand{\>}{\end{column}}
%%%%%%%%%%%%%%%%%%%%%%%%%%%%%%%%%%%%%%%%%%%%%%%%%%





% Author, Title, etc.

\title[ESNA Outlook Conference 2018]
{%
  Pipelines and Climate Change%
}

\author[Leach]
{
  Andrew~Leach
}

\institute[2018]
{
  Alberta School of Business, University of Alberta
 }

\date[10/18/2018]
{\today}


\newcommand{\degC}{$^o$C$\,$}
\newcommand{\co}{$\text{CO}_{2}\,$}
\newcommand{\tx}{$G_{2\times\text{CO}_2}$}
\newcommand{\Real}{\mathbb R}



% The main document


\begin{document}

\begin{frame}
   \tikz [remember picture,overlay]
    \node[yshift=-0.95cm,xshift=0cm] at (current page.north)
        %or: (current page.center)
        %\node[yshift=-0.75cm,xshift=4.5cm] at (current page.north west)
        %{\includegraphics[width=3in]{UA-ASB-COLOUR.png}};
        {\includegraphics[width=\paperwidth]{../../EE_logo.png}}; \vspace{1cm}
   \titlepage
   \vfill
\end{frame}



\section{Introduction}

\begin{frame}{Key Points}
\begin{itemize}
\setlength\itemsep{0 em}
\item The climate change challenges faced by Canada and the world are daunting;
\item Infrastructure is crucial for future oil sands viability, especially under low(er) prices;
\item Infrastructure demands in Canada have changed, as have prices and potential netbacks - the prize is not as big as it once was;
\item Policy changes (C-69, etc.) are not asking for the impossible on climate change tests - in fact, they're not really asking for much at all;
\item Climate change policy risk is very much like other market risks that pipelines face -- and it should be treated as such;
\item Credible climate change policy and impact assessment are necessary, if not sufficient for pipeline construction
\item Alberta, and Canada, face a much larger challenge from global action on climate change than from domestic action
\end{itemize}
\vfill
\end{frame}


\section{GHG Emissions}

\begin{frame}{Emissions across the economy}
   \tikz [remember picture,overlay]
    \node[yshift=-.75cm,xshift=0cm] at (current page.center)
        {\includegraphics[width=.9\paperwidth]{../../inventory_ghgs.png}}; \vspace{1cm}
   \vfill
\end{frame}


\begin{frame}{Targets, not policies}
   \tikz [remember picture,overlay]
    \node[yshift=-.75cm,xshift=0cm] at (current page.center)
        {\includegraphics[width=.9\paperwidth]{../../emissions_and_targets.png}}; \vspace{1cm}
   \vfill
\end{frame}

\begin{frame}{The Global Challenge is Steep}
   \tikz [remember picture,overlay]
    \node[yshift=-.75cm,xshift=0cm] at (current page.center)
        {\includegraphics[width=.9\paperwidth]{../../emissions_gap.png}}; \vspace{1cm}
   \vfill
\end{frame}


\begin{frame}{The Challenge Ahead for Oil Sands}
   \tikz [remember picture,overlay]
    \node[yshift=-.75cm,xshift=0cm] at (current page.center)
        {\includegraphics[width=.9\paperwidth]{../../inventory_ghgs.png}}; \vspace{1cm}
   \vfill
\end{frame}


\begin{frame}{The Challenge Ahead for Oil Sands}
   \tikz [remember picture,overlay]
    \node[yshift=-.75cm,xshift=0cm] at (current page.center)
        {\includegraphics[width=.9\paperwidth]{../../oil_sands_sector_ghgs.png}}; \vspace{1cm}
   \vfill
\end{frame}

\begin{frame}{The Challenge Ahead for Oil Sands}
   \tikz [remember picture,overlay]
    \node[yshift=-.75cm,xshift=0cm] at (current page.center)
        {\includegraphics[width=.9\paperwidth]{../../oil_sands_emissions_and_targets.png}}; \vspace{1cm}
   \vfill
\end{frame}




\begin{frame}{The Challenge Ahead for Oil Sands}
   \tikz [remember picture,overlay]
    \node[yshift=-.75cm,xshift=0cm] at (current page.center)
        {\includegraphics[width=.9\paperwidth]{os_ghgs_pipes.png}}; \vspace{1cm}
   \vfill
\end{frame}

\begin{frame}{The Challenge Ahead for Oil Sands}
   \tikz [remember picture,overlay]
    \node[yshift=-.75cm,xshift=0cm] at (current page.center)
        {\includegraphics[width=.9\paperwidth]{os_ghgs_no_pipes.png}}; \vspace{1cm}
   \vfill
\end{frame}



\section{Oil Infrastructure}

\begin{frame}{Not enough pipeline capacity to meet demand}
   \tikz [remember picture,overlay]
    \node[yshift=-.75cm,xshift=0cm] at (current page.center)
        {\includegraphics[width=.9\paperwidth]{../../pipe_capacity_real.png}}; \vspace{1cm}
   \vfill
\end{frame}


\begin{frame}{The market has changed since 2014 in many ways}
   \tikz [remember picture,overlay]
    \node[yshift=-.75cm,xshift=0cm] at (current page.center)
        {\includegraphics[width=.9\paperwidth]{../../pipe_capacity_new.png}}; \vspace{1cm}
   \vfill
\end{frame}


\begin{frame}{The consequences of too little pipeline capacity are now clear}
   \tikz [remember picture,overlay]
    \node[yshift=-.75cm,xshift=0cm] at (current page.center)
        {\includegraphics[width=.9\paperwidth]{macleans_2018.png}}; \vspace{1cm}
   \vfill
\end{frame}

\begin{frame}{The consequences of too little pipeline capacity are now clear}
   \tikz [remember picture,overlay]
    \node[yshift=-.75cm,xshift=0cm] at (current page.center)
        {\includegraphics[width=.9\paperwidth]{macleans_2019.png}}; \vspace{1cm}
   \vfill
\end{frame}





\begin{frame}{A Digression on PADDs}
   \tikz [remember picture,overlay]
    \node[yshift=-.75cm,xshift=0cm] at (current page.center)
        {\includegraphics[width=.9\paperwidth]{padd_map.png}}; \vspace{1cm}
   \vfill
\end{frame}


\begin{frame}{Let's not just focus on capacity as markets matter too}
   \tikz [remember picture,overlay]
    \node[yshift=-.75cm,xshift=0cm] at (current page.center)
        {\includegraphics[width=.9\paperwidth]{../../EIA_data_pulls/movements.png}}; \vspace{1cm}
   \vfill
\end{frame}




\begin{frame}{Pipelines are needed to maximize oil sands' value}
   \tikz [remember picture,overlay]
    \node[yshift=-.75cm,xshift=0cm] at (current page.center)
        {\includegraphics[width=.9\paperwidth]{../../pipe_capacity.png}}; \vspace{1cm}
   \vfill
\end{frame}



\section{Bill C-69 and Climate Tests}

\begin{frame}{What does C-69 really do?}
\begin{itemize}
\setlength\itemsep{.5em}
\item Changes the rules for major projects with respect to impact assessment;
\item Introduces the Canadian Energy Regulator (CER) which will replace the regulatory functions of the NEB;
\item Updates both impact assessment and regulatory functions to include a climate change test;
\item Makes everyone REALLY nervous.
\end{itemize}

\vfill
\end{frame}


\begin{frame}{C-69: The Canadian Energy Regulator}
      \small \vspace{-.35cm} When approving a pipeline... \\[-0.5em]
      \begin{aquote}{}
      183(2) The Commission must make its recommendation taking into account in light of among other things, any Indigenous knowledge that has been provided to the
Commission and scientific information and data all considerations that appear to it to be relevant and directly related to the pipeline including:\\[-0.5em]
\begin{itemize}\itemsep 0em
\item[a)]
the environmental effects, including any cumulative
environmental effects;
\item[f)] the availability of oil, gas (...) to the pipeline;
\item[g)] the existence of actual or potential markets
\item[j)] \textbf{the extent to which the effects of the pipeline hinder
or contribute to the Government of Canada's ability
to meet its environmental obligations and its commitments in respect of climate change};
\item[k)] any relevant assessment referred to in section 92
93 or 95 of the Impact Assessment Act; and
(l) any public interest that the Commission considers
may be affected by the issuance of the certificate or the dismissal of the application.
\end{itemize}
      \end{aquote}

\end{frame}


\begin{frame}{C-69: Governor in Council Approval}
      \small \vspace{-.35cm} When approving an impact assessment... \\[-0.5em]
      \begin{aquote}{}
      63 The Minister's determination under paragraph 10
60(1)(a) in respect of a designated project referred to in
that subsection, and the Governor in Council's determination
under section 62 in respect of a designated project
referred to in that subsection, must be based on the report
with respect to the impact assessment and a consideration of the following factors:\\[-0.5em]
\begin{itemize}\itemsep 0em
\item[a)] the extent to which the designated project contributes
to sustainability;
\item[e)] \textbf{the extent to which the effects of the designated
project hinder or contribute to the Government of
Canada's ability to meet its environmental obligations
and its commitments in respect of climate change}.;
\end{itemize}
      \end{aquote}

\end{frame}

\begin{frame}{A digression on climate change - how much insurance do you want to buy?}
   \tikz [remember picture,overlay]
    \node[yshift=-.75cm,xshift=0cm] at (current page.center)
        {\includegraphics[width=.9\paperwidth]{../../IIASA/temp_2Cprob_range_ssp2.png}}; \vspace{1cm}
   \vfill
\end{frame}


\begin{frame}{A digression on climate change - how much risk are you willing to tolerate?}
   \tikz [remember picture,overlay]
    \node[yshift=-.75cm,xshift=0cm] at (current page.center)
        {\includegraphics[width=.9\paperwidth]{../../IIASA/temp_3Cprob_range_ssp2.png}}; \vspace{1cm}
   \vfill
\end{frame}

\begin{frame}{RCP scenarios translate into temperature trajectories}
   \tikz [remember picture,overlay]
    \node[yshift=-.75cm,xshift=0cm] at (current page.center)
        {\includegraphics[width=.9\paperwidth]{../../IIASA/temps_median_SSP2.png}}; \vspace{1cm}
   \vfill
\end{frame}


\begin{frame}{Let's name the scenarios to make this easier}
   \tikz [remember picture,overlay]
    \node[yshift=-.75cm,xshift=0cm] at (current page.center)
        {\includegraphics[width=.9\paperwidth]{../../IIASA/temps_median_SSP_names2.png}}; \vspace{1cm}
   \vfill
\end{frame}

\begin{frame}{RCP scenarios translate into emissions trajectories}
   \tikz [remember picture,overlay]
    \node[yshift=-.75cm,xshift=0cm] at (current page.center)
        {\includegraphics[width=.9\paperwidth]{../../IIASA/GHG_ssp2.png}}; \vspace{1cm}
   \vfill
\end{frame}



\begin{frame}{Now, let's narrow this down to a couple of scenarios.}
   \tikz [remember picture,overlay]
    \node[yshift=-.75cm,xshift=0cm] at (current page.center)
        {\includegraphics[width=.9\paperwidth]{../../IIASA/temps_iso_median_names2.png}}; \vspace{1cm}
   \vfill
\end{frame}

\begin{frame}{Now, let's narrow this down to a couple of scenarios.}
   \tikz [remember picture,overlay]
    \node[yshift=-.75cm,xshift=0cm] at (current page.center)
        {\includegraphics[width=.9\paperwidth]{../../IIASA/temps_iso_names2.png}}; \vspace{1cm}
   \vfill
\end{frame}

\begin{frame}{What do these mean for oil demand?}
   \tikz [remember picture,overlay]
    \node[yshift=-.75cm,xshift=0cm] at (current page.center)
        {\includegraphics[width=.9\paperwidth]{../../IIASA/oil_median_iso_60_SSP2.png}}; \vspace{1cm}
   \vfill
\end{frame}


\begin{frame}{What do these mean for oil demand?}
   \tikz [remember picture,overlay]
    \node[yshift=-.75cm,xshift=0cm] at (current page.center)
        {\includegraphics[width=.9\paperwidth]{../../IIASA/oil_iso_60_SSP2.png}}; \vspace{1cm}
   \vfill
\end{frame}


\begin{frame}{Emissions policy stringency is clearly part of the picture}
   \tikz [remember picture,overlay]
    \node[yshift=-.75cm,xshift=0cm] at (current page.center)
        {\includegraphics[width=.9\paperwidth]{../../IIASA/cprice_iso_60_SSP2.png}}; \vspace{1cm}
   \vfill
\end{frame}


\begin{frame}{If you're arguing for limited GHG policy...}
   \tikz [remember picture,overlay]
    \node[yshift=-.75cm,xshift=0cm] at (current page.center)
        {\includegraphics[width=.9\paperwidth]{../../IIASA/cprice_median_iso_SSP2.png}}; \vspace{1cm}
   \vfill
\end{frame}


\begin{frame}{If you're arguing for limited GHG policy...}
   \tikz [remember picture,overlay]
    \node[yshift=-.75cm,xshift=0cm] at (current page.center)
        {\includegraphics[width=.9\paperwidth]{../../IIASA/cprice_iso_SSP2.png}}; \vspace{1cm}
   \vfill
\end{frame}






\begin{frame}{What do these tell us?}
\begin{itemize}
\setlength\itemsep{.5em}
\item Acting on climate change is going to mean significantly lower global oil demand;
\item A 2C scenario means we already should have hit peak oil demand. A 1.5C scenario implies an even more rapid decline;
\item Significant stranded asset risk will exist for long-lived assets (pipelines and oil sands production facilities) if the world acts aggressively on climate change;
\item The good news: these are the risks that our oil industry is used to dealing with, as is the NEB/CER process - we just need to let it work;
\item How do you integrate the potential for global action on climate change in the process? Ensure that market forecasts stipulate and are able to be challenged on their assumptions about climate change action.
\end{itemize}
\vfill
\end{frame}



\begin{frame}{C-69: The Canadian Energy Regulator}
      \small \vspace{-.35cm} When approving a pipeline... \\[-0.5em]
      \begin{aquote}{}
      183(2) The Commission must make its recommendation taking into account in light of among other things, any Indigenous knowledge that has been provided to the
Commission and scientific information and data all considerations that appear to it to be relevant and directly related to the pipeline including:\\[-0.5em]
\begin{itemize}\itemsep 0em
\item[a)]
the environmental effects, including any cumulative
environmental effects;
\item[f)] \textbf{the availability of oil, gas (...) to the pipeline;}
\item[g)] \textbf{the existence of actual or potential markets}
\item[j)]the extent to which the effects of the pipeline hinder
or contribute to the Government of Canada's ability
to meet its environmental obligations and its commitments in respect of climate change;
\item[k)] any relevant assessment referred to in section 92
93 or 95 of the Impact Assessment Act; and
(l) any public interest that the Commission considers
may be affected by the issuance of the certificate or the dismissal of the application.
\end{itemize}
      \end{aquote}

\end{frame}

\begin{frame}{Can oil sands and pipelines pass a climate test? Maybe.}
  \tikz [remember picture,overlay]
    \node[yshift=-0.5cm,xshift=0cm] at (current page.center)
        {\includegraphics[width=.8\paperwidth]{results3.png}}; \vspace{1cm}
\vfill
\end{frame}


\begin{frame}{Can oil sands and pipelines pass a climate test? Maybe.}
  \tikz [remember picture,overlay]
    \node[yshift=-0.5cm,xshift=0cm] at (current page.center)
        {\includegraphics[width=.8\paperwidth]{results3b.png}}; \vspace{1cm}
\vfill
\end{frame}

\begin{frame}{Can oil sands and pipelines pass a climate test? Maybe.}
  \tikz [remember picture,overlay]
    \node[yshift=-0.5cm,xshift=0cm] at (current page.center)
        {\includegraphics[width=.8\paperwidth]{../../oil_sands_emissions_and_targets.png}}; \vspace{1cm}
\vfill
\end{frame}


\section{Conclusion}

\begin{frame}{Conclusion}
\begin{itemize}
\setlength\itemsep{.5em}
\item The threat of climate change and the potential value of oil sands present Canada with very challenging choices;
\item Oil sands projects will, definitely be higher with pipelines than without;
\item Does that mean that global emissions will, definitely, be higher with pipelines than without? No.;
\item Serious global action on climate change will, almost assuredly, mean no further oil sands expansion and (dis)orderly wind-down of the oil sands industry over decades;
\item Global action, not domestic policies, will have the largest impacts on oil sands revenues;
\item The good news, again: these are the risks that our oil industry is used to dealing with, as is the NEB/CER process. Let it work - don't tie its hands.
\end{itemize}
\vfill
\end{frame}





\begin{frame}{Contact info}
\begin{center}
Andrew Leach\bigskip \\
School of Business, University of Alberta\bigskip \\
Email: \href{mailto:aleach@ualberta.ca}{andrew.leach@ualberta.ca}\bigskip \\
Twitter: \href{http://twitter.com/andrew_leach}{\url{@andrew_leach}}
\end{center}
\vfill
\end{frame}



\begin{frame}{If not oil sands then what?}
  \tikz [remember picture,overlay]
    \node[yshift=-0.5cm,xshift=0cm] at (current page.center)
        {\includegraphics[width=.8\paperwidth]{cali_crude_tab.png}}; \vspace{1cm}
\vfill
\end{frame}

\begin{frame}{If not oil sands then what?}
  \tikz [remember picture,overlay]
    \node[yshift=-0.5cm,xshift=0cm] at (current page.center)
        {\includegraphics[width=.8\paperwidth]{masnadi.png}}; \vspace{1cm}
\vfill
\end{frame}


\begin{frame}{If not oil sands then what?}
  \tikz [remember picture,overlay]
    \node[yshift=-0.5cm,xshift=0cm] at (current page.center)
        {\includegraphics[width=.8\paperwidth]{masnadi_fig_1.png}}; \vspace{1cm}
\vfill
\end{frame}





\end{document} 